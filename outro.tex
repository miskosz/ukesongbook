% Adjust formatting.
\begingroup % Original formatting is reset back with \endgroup
\setlength{\parindent}{0pt}
\setlength{\parskip}{\baselineskip}
\large
\def\arraystretch{1.5}%


\section*{Okienko z hudobnej teórie: Značenie akordov}

Je pravda, že značky, ktoré sa používajú pre akordy, môžu byť mätúce. Hudobné
značenie za vyvíjalo počas mnohých storočí spolu s hudbou, a preto nie je
vždy logické. V tejto sekcii si vysvetlíme, ako sa značenie odvodzuje.

Značenie akordov vychádza s jazzu, a preto je v angličtine. Predtým noty určovali
presne, aké noty má hráč hrať. V jazze sa kladie dôraz na improvizáciu a preto vznikol
zápis pre harmónie (akordy).

Musíme ale začať od základu.


\subsection*{Tóny}

Pravidelné kmitanie vzduchu vnímame ako \textbf{tón} a frekvenciu tohto kmitania ako
\textbf{výšku tónu.} Naše ucho vníma frekvencie medzi 20-20000 Hz. Je nespočetne veľa
frekvencií ktoré sa dajú zahrať, hudba v našej (západnej) kultúre používame iba niektoré
z nich.

Vzdialenosť výšok dvoch tónov sa nazýva \textbf{interval.} Naše ucho vníma frekvencie \textit{logaritmicky},
t.j. vzdialenosť medzi tónmi nám príde rovnaká ak je pomer ich frekvencií rovnaký. Jednoduché pomery nám
znejú súznejúco. Najjednoduchší a najviac súznejúci pomer 2:1 nazývame \textbf{oktáva}.

Tón s frekvenciou 440 Hz nazývame \textbf{komorné A} alebo A\textsubscript{4}. Tóny líšiace sa
o oktávu označujeme rovnakým písmenom, takže napríklad frekvencie 220 Hz, 880 Hz alebo 1760 Hz sú tiež tóny A
(konkrétne A\textsubscript{3}, A\textsubscript{5} a A\textsubscript{6}). Jednu oktávu delíme na
\textbf{12 rovnako veľkých intervalov} nazvaných \textbf{poltón} a jednotlivé tóny označujeme nasledovne:

\begin{center}
\begin{tabular}{ C{2em} C{2em} C{2em} C{2em} C{2em} C{2em} C{2em} C{2em} C{2em} C{2em} C{2em} C{2em} }
    0 & 1   & 2 & 3 & 4   & 5 & 6   & 7 & 8 & 9   & 10 & 11 \\
    \hline
    A & A\# & B & C & C\# & D & D\# & E & F & F\# & G & G\# \\
      & Bb  &   &   & Db  &   & Eb  &   &   & Gb  &   & Ab  \\
\end{tabular}
\end{center}

Toto na prvý pohľad mätúce značenie je odvodené od siedmych tónov stupnice A moll:\footnote{U nás sa označenie noty B skomolilo na H, v tomto texte používame logickejšie anglické značenie.}

\begin{center}
\begin{tabular}{ C{2em} C{2em} C{2em} C{2em} C{2em} C{2em} C{2em} }
    A & B & C & D & E & F & G \\
\end{tabular}
\end{center}

Týchto sedem tónov je rovnakých ako sedem tónov stupnice C dur, a sú to práve biele klávesy na klavíri.
V minulosti hudba a nástroje používali iba tieto tóny. Tie ale nemajú medzi sebou rovnomerné
intervaly, dvojice B-C a E-F sú bližšie. Časom vzniklo značenie pre chýbajúce tóny,
\textbf{krížik `\#`} označuje zvýšenie a \textbf{béčko `b`} zníženie o poltón. Niektoré tóny tak majú
viaceré názvy.

\textit{%
Na ukulele alebo gitare si môžete všimnúť, že dvanásty pražec je umiestnený presne v polovici struny.
Struna skrátena na polovicu totiž kmitá dvakrát rýchlejšie, a teda na dvanástom pražci nájdete tón
o oktávu vyšsie než je tón struny. Každý z dvanásť pražcov medzi prislúcha k jednému z dvanástich tónov
oktávy.
}


\subsection*{Intervaly}


\begin{itemize}
    \item Tabulka: num poltonov, sk nazov, en nazov, add notacia
    \item Odvodene od durovej stupnice
\end{itemize}

\begin{tabular}{ c c c c c }
    \# poltónov & sk & en & tón od C & add \\ 
    \hline
    0 & príma & unison & C & \\ 
    1 & malá sekunda & minor 2nd & C\# & \\ 
    2 & veľká sekunda & major 2nd & D & add2 \\ 
    3 & malá tercia & minor 3rd & D\# & \\ 
    4 & veľká tercia & major 3rd & E & \\ 
    5 & kvarta & 4th & F & add4 \\ 
    6 & -- & tritone & F\# & \\ 
    7 & kvinta & 5th & G & \\ 
    8 & malá sexta & minor 6th & G\# & \\ 
    9 & veľká sexta & major 6th & A & 6 \\ 
    10 & malá septima & minor 7th & A\# & 7 \\ 
    11 & veľká septima & major 7th & B & maj7 \\ 
    12 & oktáva & octave & C & \\ 
\end{tabular}


Akordy

\begin{itemize}
    \item Suzvuk viacerych tonov
    \item Zavisi/nezavisi na oktave
    \item Obraty akordu
    \item Co sa deje na ukulele (nahodne podadie a oktavy)
\end{itemize}

Akordy: Trojzvuky

\begin{itemize}
    \item Zapadna hudba: skladame na seba tercie
    \item Kvintakord
    \item Durovy, mollovy
    \item Modifikacie
    \item Dim, aug
    \item sus
\end{itemize}

Akordy: Stvorzvuky a ine

\begin{itemize}
    \item 7, m7
    \item maj7, mmaj7
    \item dim7, m7b5
\end{itemize}

Akordy: Power chords and slash chords

\begin{tabularx}{\linewidth}{ c c c c c X }
    \# poltónov & sk & en & tón od C & add & w \\ 
    \hline
      & major & P1-M3-P5 & 0-4-7 & C E G & Blah \\
    m & major & P1-m3-P5 & 0-3-7 & C Eb G & Blah \\
    N/A & diminished & P1-m3-d5 & 0-3-6 & C Eb Gb & Blah \\
    aug & augmented & P1-M3-A5 & 0-4-8 & C E G\# & Blah \\
    sus2 & suspended 2nd & P1-M2-P5 & 0-2-7 & C D G & Blah \\
    sus4 & suspended 4th & P1-P4-P5 & 0-5-7 & C F G & Blah \\
    7 & dominant 7th & -- & 0-4-7-10 & C E G Bb & Blah \\
    m7 & minor 7th & -- & 0-3-7-10 & C Eb G Bb & Blah \\
    maj7 & major 7th & -- & 0-4-7-11 & C E G B & Blah \\
    mmaj7 & minor major 7th & -- & 0-3-7-11 & C Eb G Bb & Blah \\
    dim & diminished 7th & -- & 0-3-6-9 & C Eb Gb A & Blah \\
    m7b5 & half-diminished & -- & 0-3-6-10 & C Eb Gb Bb & Blah \\
\end{tabularx}

\endgroup