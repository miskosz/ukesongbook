% Adjust formatting.
\begingroup % Original formatting is reset back with \endgroup
\setlength{\parindent}{0pt}
\setlength{\parskip}{\baselineskip}
\large
\def\arraystretch{1.5}%


\section*{Okienko z hudobnej teórie: Značenie akordov}

Ak ťa vždy zaujímalo, čo vlastne znamená to \emph{maj7} v \emph{Cmaj7}, alebo ako z názvu akordu
odvodím, kde mám chytiť struny, táto sekcia je pre teba. Varovanie na úvod, keďže muzické značenie
sa vyvíjalo stáročia, môže byť miestami mätúce. Je však za ním istá logika.

Súčasné značenie akordov sa začalo používať s nástupom jazzu. Dovtedy notový zápis presne určoval,
ktoré noty treba hrať. S akordami to tak nie je, jeden akord sa dá zahrať viacerými spôsobmi.
Môžeš si všimnúť, že názvy akordov používajú angličtinu -- je to preto, že jazz vznikol v Spojených Štátoch.


\subsection*{Tóny}

Pravidelné kmitanie vzduchu vnímame ako \textbf{tón} a frekvenciu tohto kmitania ako
\textbf{výšku tónu.} Naše ucho vníma frekvencie medzi 20-20000 Hz. Je nespočetne veľa
frekvencií ktoré sa dajú zahrať, hudba našej (západnej) kultúry používa iba niektoré
z nich.

Vzdialenosť výšok dvoch tónov sa nazýva \textbf{interval.} Naše ucho vníma frekvencie \textit{logaritmicky},
t.j. vzdialenosť medzi tónmi nám príde rovnaká ak je pomer ich frekvencií rovnaký. Jednoduché pomery nám
znejú súznejúco. Najjednoduchší a najviac súznejúci pomer 2:1 nazývame \textbf{oktáva}.

Tón s frekvenciou 440 Hz nazývame \textbf{komorné A} alebo A\textsubscript{4}. Tóny líšiace sa
o oktávu označujeme rovnakým písmenom, takže napríklad frekvencie 220 Hz, 880 Hz alebo 1760 Hz sú tiež tóny A
(konkrétne A\textsubscript{3}, A\textsubscript{5} a A\textsubscript{6}). Jednu oktávu delíme na
\textbf{12 rovnako veľkých intervalov} nazvaných \textbf{poltón} a jednotlivé tóny označujeme nasledovne:

\begin{center}
\begin{tabular}{ C{2em} C{2em} C{2em} C{2em} C{2em} C{2em} C{2em} C{2em} C{2em} C{2em} C{2em} C{2em} }
    0 & 1   & 2 & 3 & 4   & 5 & 6   & 7 & 8 & 9   & 10 & 11 \\
    \hline
    A & A\# & B & C & C\# & D & D\# & E & F & F\# & G & G\# \\
      & Bb  &   &   & Db  &   & Eb  &   &   & Gb  &   & Ab  \\
\end{tabular}
\end{center}

Toto na prvý pohľad mätúce značenie je odvodené od siedmych tónov stupnice A moll:\footnote{U nás sa označenie noty B skomolilo na H, v tomto texte používame logickejšie anglické značenie.}

\begin{center}
\begin{tabular}{ C{2em} C{2em} C{2em} C{2em} C{2em} C{2em} C{2em} }
    A & B & C & D & E & F & G \\
\end{tabular}
\end{center}

Týchto sedem tónov je rovnakých, ako sedem tónov stupnice C dur, a sú to práve biele klávesy na klavíri.
V minulosti hudba a nástroje používali iba tieto tóny. Tie ale nemajú medzi sebou rovnomerné
intervaly, dvojice B-C a E-F sú bližšie. Časom vzniklo značenie pre chýbajúce tóny,
\textbf{krížik `\#`} označuje zvýšenie a \textbf{béčko `b`} zníženie o poltón. Niektoré tóny tak majú
viaceré názvy.

\textit{%
Na ukulele alebo gitare si môžeš všimnúť, že dvanásty pražec je umiestnený presne v polovici struny.
Struna skrátena na polovicu totiž kmitá dvakrát rýchlejšie, a teda na dvanástom pražci nájdeš tón
o oktávu vyšsie, než je základný tón struny. Každý z dvanásť pražcov medzi prislúcha k jednému
z dvanástich tónov oktávy.
}


\subsection*{Intervaly}

Názvy intervalov sú odvodené od durovej stupnice. Tá pozostáva zo siedmych tónov, nazývaných
v kontexte stupnice \textbf{stupne}, ktoré sú od základného tónu vzdialené 0, 2, 4, 5, 7, 9 a 11
poltónov. Pozri tabuľku nižšie pre základné názvy intervalov.

Intervaly príma, kvarta, kvinta a oktáva sa používajú s prívlastkom \textbf{čistá} \textit{(perfect)}.
Sekunda, tercia, sexta a septima môžu byť \textbf{malé} \textit{(minor)} alebo \textbf{veľké}
\textit{(major)}. V durovej stupnici sa vyskytujú iba čisté a veľké intervaly.

Okrem čistých, malých a veľkých môžu byť intervaly \textbf{zmenšené} \textit{(diminished)} alebo
\textbf{zväčšené} \textit{(augmented)}. Zmenšený interval je o poltón menší ako jeho čistá alebo
malá verzia, naopak zväčšený interval je o poltón väčší ako jeho čistá alebo veľká verzia.

\textit{%
Intervaly môžu mať viacero názvov. Napríklad zmenšená tercia je veľká sekunda (2 poltóny) alebo
zmenšená kvinta je zväčšená kvarta (6 poltónov).
}



\begin{center}
\begin{tabular}{ c c c c c }
    \# poltónov & sk & en & interval od C \\
    \hline
    0 & čistá príma & perfect unison & C \\
    1 & malá sekunda & minor 2nd & C\# \\
    2 & veľká sekunda & major 2nd & D \\
    3 & malá tercia & minor 3rd & D\# \\
    4 & veľká tercia & major 3rd & E \\
    5 & čistá kvarta & perfect 4th & F \\
    6 & -- & tritone & F\# \\
    7 & čistá kvinta & perfect 5th & G \\
    8 & malá sexta & minor 6th & G\# \\
    9 & veľká sexta & major 6th & A \\
    10 & malá septima & minor 7th & A\# \\
    11 & veľká septima & major 7th & B \\
    12 & čistá oktáva & perfect octave & C \\
\end{tabular}
\end{center}


\subsection*{Akordy}

Akord je súzvuk viacerých tónov. Zápis akordu pozostáva z dvoch častí. Prvá časť
určuje \textbf{základný tón} akordu (ako napríklad C alebo A\#). Druhá časť (napríklad 7 alebo sus4)
určuje \textbf{intervaly ostatných tónov}, ktoré sa majú hrať so základným tónom.

V základnej forme sa akord hrá presne s uvedenými intervalmi. Pri hre na ukulele, aby sme využili
všetky struny a aby sa akord dal chytiť, sa niektoré tóny môžu zdvojiť alebo presunúť do inej
oktávy. Takéto prevedenie akordu sa anglicky nazýva \textit{chord voicing}.

\textit{%
Napríklad mollový akord okrem základného tónu obsahuje tón o malú terciu a tón o čistú kvintu vyššie. Preto napríklad v akorde Dm sú tóny D, F a A. Nasledujúce
dva \uv{voicings} pre ukulele pozostávajú z tónov A\textsubscript{4} D\textsubscript{4} F\textsubscript{4} A\textsubscript{4}
a D\textsubscript{5} F\textsubscript{4} A\textsubscript{4} D\textsubscript{5}.
}

\begin{center}
\ukechord{Dm} \quad \ukechord{Dm_fret5}
\end{center}

Akordy sa typicky skladajú z tónov v rozostupoch o malú alebo veľkú terciu. V prípade troch tónov sú štyri možnosti,
ako takto akord vytvoriť. Tieto štyri základné \textbf{trojzvuky} \textit{(triads)} sa nazývajú
\textbf{durový} \textit{(major)}, \textbf{mollový} \textit{(minor)}, \textbf{zväčšený} \textit{(augmented)}
a \textbf{zmenšený} \textit{(diminished)}.

Základ západnej hudby tvoria durový a mollový akord, ktoré obidva obsahujú terciu a čistú kvintu. Modifikáciou
tercie vznikajú \textbf{\textit{suspended}} akordy a jej vynechaním \textbf{\textit{power chord.}}

Najbežnejšie štvortónové akordy vznikajú pridaním septimy k niektorému zo základných trojzvukov.

Nasledujúca tabuľka obsahuje prehľad najbežnejších akordov spolu s intervalmi, ktoré obsahujú.
Popisy sú uvedené po anglicky, aby lepšie približovali význam značky. Tento výčet akordov zďaleka
nie je úplný, existujú ďalšie, menej bežné označenia, ktoré tu neuvádzame.

{\smaller
\begin{tabularx}{\linewidth}{ l l l l X }
    značka & anglický názov & intervaly* & akord od C & popis \\
    \hline
      & major & 0-4-7 & C E G & Major 3rd and perfect 5th \\
    \textbf{m} & minor & 0-3-7 & C Eb G & Minor 3rd and perfect 5th \\
    \textbf{aug} & augmented & 0-4-8 & C E G\# & Major 3rd and augmented 5th \\
    N/A & diminished & 0-3-6 & C Eb Gb & Minor 3rd and diminished 5th. Usually diminished 7th chord is played instead. \\
    \hline
    \textbf{sus2} & suspended 2nd & 0-2-7 & C D G & Major 2nd and perfect 5th  \\
    \textbf{sus4} & suspended 4th & 0-5-7 & C F G & Perfect 4th and perfect 5th \\
    \textbf{5} & power chord & 0-7 & C G & Perfect 5th only. Popular in rock. \\
    \hline
    \textbf{7} & dominant 7th &  0-4-7-10 & C E G Bb & Major triad with minor 7th  \\
    \textbf{maj7} & major 7th &  0-4-7-11 & C E G B & Major triad with major 7th \\
    \textbf{m7} & minor 7th &  0-3-7-10 & C Eb G Bb & Minor triad with minor 7th \\
    \textbf{mmaj7} & minor major 7th &  0-3-7-11 & C Eb G B & Minor triad with major 7th \\
    \textbf{dim} & diminished 7th &  0-3-6-9 & C Eb Gb A & Diminished triad with diminished 7th \\
    \textbf{m7b5} & half-diminished &  0-3-6-10 & C Eb Gb Bb & Diminished triad with minor 7th. Also called ''minor 7th flat 5''. \\
\end{tabularx}
}
{\smaller *Intervaly v poltónoch od základného tónu.}

\bigskip

\textit{%
Jazzové akordy často obsahú aj viac než štyri tóny, ako napríklad nónové alebo undecimové akordy (9th and 11th chords). Ako sa tie hrajú na štvorstrunovom ukulele? Niektoré tóny sa musia jednoducho vynechať. Ktoré závisí na interpretácii, ale často to zvykne byť kvinta alebo základný tón akordu.
}

\textit{%
Spomeňme ešte, že \textbf{add} označuje pridanie intervalu do akordu. Napríklad Gmadd4 obsahuje okrem G, Bb a D z akordu Gm tiež tón C, ktorý je od G vzdialený o kvartu.
}

\subsection*{Akordy s lomítkom}

\textbf{Akordy s lomítkom} \textit{(slash chords)} pozostávajú z názvu akordu, lomítka, a názvu tónu
-- napríklad Hm/D alebo C/B, čítaj \uv{nad} \textit{(\uv{over})}. Tento zápis inštruuje, že do akordu
sa má pridať (ak tam ešte nie je) tón za lomítkom, a má to byť ten najhlbšie hraný tón. Typicky sa hrá v basovom registri.

Na rozdiel od klavíra alebo gitary však ukulele nemá široký rozsah, a preto je na ňom problematické
hrať akordy s lomítkom. Vo všeobecnosti sa preto v ukulele spevníkoch týmto akordom vyhýba,
alebo sú nahradzované (napríklad Cmaj7 namiesto C/B a podobne).

\textit{%
Niekedy sa však tento zápis pre ukulele \uv{zneužíva} na odlíšenie rôznych \uv{chord voicings}.
Napríklad Am/C niekedy označuje nasledujúci \uv{voicing} akordu A moll so zdvojeným C.
}

\begin{center}
\ukechord{Am/C}
\end{center}


\endgroup