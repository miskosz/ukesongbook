% From https://tex.stackexchange.com/questions/318670/generating-ukulele-chord-diagrams

\newcommand\drawukulelechord[3][]{%
  \edef\chord{#3}%
  \begin{tikzpicture}[x=2ex,y=2ex,line cap=butt,line width=.4pt,
    baseline=(current bounding box.center),#1]
    \draw[line width=1pt] (1,0) -- (4,0);
    \foreach \d in {1,...,4}{\draw (1,-\d) -- (4,-\d);}
    \foreach \d[count=\p] in \chord {
      \draw (\p,0) -- (\p,-4.5);
      \ifdefstring{\d}{-1}{
        \draw (\p,.25) +(-.125,-.125) -- +(.125,.125)
        +(-.125,.125) -- +(.125,-.125);
      }{
        \ifdefstring{\d}{0}{
          % \draw (\p,.25) circle(.125); % opened string
        }{
          \fill (\p,.5-1*\d) circle(.25);
        }
      }
    }
    \node at (2.5,-5.5) {#2};
    \path[use as bounding box] (0.5,1.0) rectangle (4.5,-5);
    % \path[use as bounding box] (0.5,.5) rectangle (4.5,-5); % TODO: make larger now that I added the text?
  \end{tikzpicture}%
}
\makeatletter
\newcommand\defineukulelechord[3][\@nil]{%
  \csdef{@ukulelechord@#2}{%
    \def\tmp{#1}%
    \ifx\tmp\@nnil
       \drawukulelechord{#2}{#3}
    \else
       \drawukulelechord{#1}{#3}
    \fi%
  }%
}
\newcommand\ukulelechord[1]{%
  \ifcsdef{@ukulelechord@#1}{%
    \csuse{@ukulelechord@#1}%
  }{%
    \GenericError{}{Undefined ukulele chord '#1'}{}{}% 
  }%
}
\makeatother

\newcommand{\ukechord}{\ukulelechord}

% A
\defineukulelechord{A}{2,1,0,0}
\defineukulelechord{Am}{2,0,0,0}
\defineukulelechord{Asus2}{2,2,0,0}
\defineukulelechord{A7}{0,1,0,0}
\defineukulelechord{A7sus4}{0,2,0,0}

% Bb
\defineukulelechord{Bb}{3,2,1,1}

% B

% C
\defineukulelechord{C}{0,0,0,3}
\defineukulelechord[C]{Cfret3}{5,4,3,3}
\defineukulelechord{C7}{0,0,0,1}
\defineukulelechord{Cmaj7}{0,0,0,2}

% C#
\defineukulelechord[C\#dim]{Csharpdim}{0,1,0,1}

% D
\defineukulelechord{D}{2,2,2,0}
\defineukulelechord{Dm}{2,2,1,0}
\defineukulelechord{Dm7}{2,2,1,3}

% D#
\defineukulelechord[D\#dim]{Dsharpdim}{2,3,2,3}

% E
\defineukulelechord{Em}{0,4,3,2}
\defineukulelechord{Eb}{0,3,3,1}
\defineukulelechord{E7}{1,2,0,2}

% F
\defineukulelechord{F}{2,0,1,0}
\defineukulelechord{F7}{2,3,1,0}
\defineukulelechord[F7]{F7alt}{2,3,1,3}
\defineukulelechord{Fdim}{1,2,1,2}

%G
\defineukulelechord{G}{0,2,3,2}
\defineukulelechord{G7}{0,2,1,2}
\defineukulelechord{Gm}{0,2,3,1}
\defineukulelechord{Gm7}{0,2,1,1}

