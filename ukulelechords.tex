% From https://tex.stackexchange.com/questions/318670/generating-ukulele-chord-diagrams

% Usage: \drawukulelechord[StartFret]{Label}{g,c,e,a}
% Where g,c,e,a are fret numbers, or -1 for a muted string.
\newcommand\drawukulelechord[3][1]{%
  \edef\startfret{#1}% default=1
  \edef\chordlabel{#2}%
  \edef\frets{#3}%
  \begin{tikzpicture}[
    x=2ex,
    y=2ex,
    line cap=butt,
    line width=.4pt,
    baseline=(current bounding box.center)
  ]
    \ifnum \startfret=1
      \draw[line width=1pt] (1,0) -- (4,0);
      \def\topy{0}
    \else
      \def\topy{0.5}
      \draw (1,0) -- (4,0);
      \node at (4.625,-0.5) {\smaller\smaller \startfret};
    \fi%
    \foreach \d in {1,...,4}{\draw (1,-\d) -- (4,-\d);}
    \foreach \d[count=\p] in \frets {
      \draw (\p,\topy) -- (\p,-4.5);
      \ifdefstring{\d}{-1}{
        \draw (\p,.25) +(-.125,-.125) -- +(.125,.125)
        +(-.125,.125) -- +(.125,-.125);
      }{
        \ifdefstring{\d}{0}{
          % \draw (\p,.25) circle(.125); % open string
        }{
          \fill (\p,.5-1*\d+\startfret-1) circle(.25);
        }
      }
    }
    \node at (2.5,-5.5) {\chordlabel};
    \path[use as bounding box] (0.5,1.0) rectangle (4.5,-5);
    % \path[use as bounding box] (0.5,.5) rectangle (4.5,-5); % TODO: make larger now that I added the text?
  \end{tikzpicture}%
}
\makeatletter
\newcommand\defineukulelechord[3][\@nil]{%
  \csdef{@ukulelechord@#2}{%
    \def\chordlabel{#1}%
    \def\chordid{#2}%
    \def\frets{#3}%
    \ifx\chordlabel\@nnil
      \drawukulelechord{\chordid}{\frets}
    \else
      \drawukulelechord{\chordlabel}{\frets}
    \fi%
  }%
}
\newcommand\defineukulelechordfret[4]{%
  \csdef{@ukulelechord@#2}{%
    \def\chordlabel{#1}%
    \def\chordid{#2}%
    \def\startfret{#3}%
    \def\frets{#4}%
    \drawukulelechord[\startfret]{\chordlabel}{\frets}
  }%
}
\newcommand\ukulelechord[1]{%
  \ifcsdef{@ukulelechord@#1}{%
    \csuse{@ukulelechord@#1}%
  }{%
    \GenericError{}{Undefined ukulele chord '#1'}{}{}% 
  }%
}
\makeatother

\newcommand{\ukechord}{\ukulelechord}

% Ab
\defineukulelechord{Ab}{5,3,4,3}

% A
\defineukulelechord{A}{2,1,0,0}
\defineukulelechord{Am}{2,0,0,0}
\defineukulelechord{Am7}{0,0,0,0}
\defineukulelechord{Am7b5}{2,3,3,3}
\defineukulelechord{Asus4}{2,2,0,0}
\defineukulelechord{A7}{0,1,0,0}
\defineukulelechord{A7sus4}{0,2,0,0}

% A#
\defineukulelechord[A\#m]{Asharpm}{3,1,1,1}

% Bb
\defineukulelechord{Bb}{3,2,1,1}
% \defineukulelechordfret{Bb}{Bbfret5}{5}{7,5,6,5}

% B
\defineukulelechord{B7}{2,3,2,2}
\defineukulelechord{Bm}{4,2,2,2}
\defineukulelechord{Bm7}{2,2,2,2}
\defineukulelechord{Bdim}{1,2,1,2}

% C
\defineukulelechord{C}{0,0,0,3}
\defineukulelechord[C]{Cbarre3}{5,4,3,3}
\defineukulelechord{C7}{0,0,0,1}
\defineukulelechord{C6}{0,0,0,0}
\defineukulelechord{Cmaj7}{0,0,0,2}
\defineukulelechord{Cm}{0,3,3,3}
\defineukulelechord{Cm7}{3,3,3,3}
\defineukulelechord{Cm6}{2,3,3,3}  % Cm/A

% C#
\defineukulelechord[C\#]{Csharp}{1,1,1,4}
\defineukulelechordfret{C\#}{Csharpfret4}{4}{6,5,4,4}
\defineukulelechord[C\#dim]{Csharpdim}{0,1,0,1}

% D
\defineukulelechord{D}{2,2,2,0}
\defineukulelechord[D]{Dbarre2}{2,2,2,5}
\defineukulelechord{D7}{2,2,2,3}
\defineukulelechord{Dm}{2,2,1,0}
\defineukulelechord{Dm7}{2,2,1,3}
\defineukulelechord{Dsus2}{2,2,0,0}
\defineukulelechord{Dsus4}{2,2,3,0}
\defineukulelechord{Daug}{3,2,2,5}

% D#
\defineukulelechord[D\#dim]{Dsharpdim}{2,3,2,3}
\defineukulelechord[D\#m]{Dsharpm}{3,3,2,1}
\defineukulelechord[D\#m7]{Dsharpm7}{3,3,2,4}

% Eb
\defineukulelechord{Eb}{0,3,3,1}

% E
\defineukulelechord{E}{4,4,4,2}
\defineukulelechord{Em}{0,4,3,2}
\defineukulelechord{Em7}{0,2,0,2}
\defineukulelechord{E7}{1,2,0,2}

% F
\defineukulelechord{F}{2,0,1,0}
\defineukulelechord{F7}{2,3,1,0}
\defineukulelechord[F7]{F7alt}{2,3,1,3}
\defineukulelechord{Fm}{1,0,1,3}
\defineukulelechord{Fdim}{1,2,1,2}
\defineukulelechord{Fadd2}{0,0,1,0}

% F#
\defineukulelechord[F\#]{Fsharp}{3,1,2,1}
\defineukulelechord[F\#m]{Fsharpm}{2,1,2,0}
\defineukulelechord[F\#dim]{Fsharpdim}{2,3,2,3}


% G
\defineukulelechord{G}{0,2,3,2}
\defineukulelechord{G7}{0,2,1,2}
\defineukulelechord{Gm}{0,2,3,1}
\defineukulelechord{Gm7}{0,2,1,1}
\defineukulelechord{Gm6}{0,2,0,1}
\defineukulelechord{Gmmaj7}{0,2,2,1}
\defineukulelechord{Gsus2}{0,2,3,0}

% G#
\defineukulelechord[G\#dim]{Gsharpdim}{1,2,1,2}
