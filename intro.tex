% Adjust formatting.
\begingroup % Original formatting is reset back with \endgroup
\setlength{\parindent}{0pt}
\setlength{\parskip}{\baselineskip}
\large

\section*{Úvod}

Všetkých Vás vítam v Citrónovom ukulele spevníku! Či už hráte na ukulele alebo iný nástroj,
a či ste začiatočníci alebo pokročilí, dúfam, že si v spevníku nájdete niečo pre Vás.

Tento spevník vznikol hlavne z dvoch dôvodov. Prvým je krátkodobá pamäť. U nás sa v spevníkoch
zvyknú uvádzať akordy iba v prvej slohe a v ostaných si ich treba buď pamätať alebo skákať očami
hore-dole, čo mne nikdy nešlo. Tu sú akordy uvedené všade.

Druhý dôvod sú piesne, ktoré poznám iba približne. Možno sa Vám už stalo, že zistíte, že neviete
ako presne vybranú pieseň zahrať, pretože vlastne poznáte len jej refrén alebo časť slohy.
Alebo sprevádzate niekoho, kto ju pozná, akurát by ste potrebovali vedieť, kedy zmeniť akord.
V tomto spevníku je preto naznačené, ako dlho akordy hrať.

A keďže hrám na ukulele, tak vznikol ukulele spevník.\footnote{Spevník bol inšpirovaný Londýnskym
spevníkom Ukulele Wednesdays. Odporúčam :)} Akordy sú miestami prispôsobé tak, aby sa dobre hrali
na tomto malebnom nástroji, ale spevník je použiteľný aj pre napríklad gitaru. Tiež dúfam,
že takto môžem prispieť k formujúcej sa ukulele komunite na Slovensku.

V neposlednom rade, uživajte si radosť z hrania! Neexistuje nič také ako \uv{ako sa pesnička má hrať
správne}, je na každom interpretovi, ako si ju prispôsobí, aby sa mu páčila. Ja som sa snažil
akordy uvádzať blízko originálu, ale nakoľko je to veľmi rozsiahly projekt, určite sú v spevníku chyby.
(Akordy, text, časovanie.) Pokiaľ na nejakú narazíte, alebo by ste chceli do spevníku prispieť, dajte
mi určite vedieť! [TODO: Štylistika.]

Z lockdownu v Londýne,

\textit{%
Mišo \\
Január 2021
}

\section*{Použité značky}

\subsection*{Akordy}

V spevníku používame anglické značenie, ktoré je bežné u zahraničných piesní. Zápis \textbf{B, Bb}
preto označuje noty \textbf{H, B} v nemeckom značení, ktoré sa učí u nás.

Hudobný zápis je organizovaný do \textbf{taktov.} Každý takt pozostáva z niekoľkých
\textbf{dôb.} V spevníku značka \ch{C} značí akord, ktorý trvá celý takt, \ch{C\beats2}
značí akord trvajúci toľko dôb, koľko je čiarok. Na začiatku piesne je uvedené zápisom \beatsperchord{2},
koľko dôb má jeden takt. Keďže väčšina piesní sa hrá na štyri doby, predpis \beatsperchord{4} vynechávame.

Text obsahuje diagramy akordov pre štandardné ukulele ladenie GCEA. Pre úsporu miesta sú niekedy
vynechané nasledujúce štyri základné akordy:
\begin{center}
\smaller
\ukechord{C}
\ukechord{G}
\ukechord{F}
\ukechord{Am}
\larger
\end{center}

Na rozdiel od ostatných spevníkov sú akordy uvedené priamo v texte, text sa tak lepšie zmestí
na stránku. Pokiaľ medzi akordom a textom \textbf{nie je medzera,} slová textu sa začínajú spievať
naraz s akordom, s medzerou sa text začína spievať neskôr.\footnote{Minimálne taká bola snaha :)}

\textit{%
Poznámka pre pokročilých: Pretože ukulele nemá basové struny, akordy s lomítkom sa všeobecne
nepoužívajú. Základný tón akordu preto nemusí byť rovnaký ako čo má hrať basgitara. Napríklad
(C/B) alebo (D/A) môžu byť prepísané ako (Cmaj7) alebo (D).
}

\medskip

\begin{tabularx}{\linewidth}{ l X }
    \ch{C\beats2} & Hraj akord toľko dôb koľko je čiarok \\ 
    \ch{C} & Hraj akord na dĺžku jedného taktu \\
    \ch{C\rep2} & Hraj akord na dĺžku viacerých taktov \\
    \ch{*C} & Skráť predošlý akord a predĺž tento o dobu alebo poldoby (podľa piesne)\footnote{%
    Napríklad \ch{D} \ch{*G} môže byť realizované ako \ch{D\beats3}\ch{G\beats5}.} \\
    \ch{C-G-C} & Skratka pre hranie akordov aby sa zmestili do jedného taktu \\
    \ch{N.C.} & Hudobná pauza, z anglického \textit{no chord}
\end{tabularx}


\subsection*{Rytmický sprievod}

Rukou, ktorá nedrží akord, hráme \textbf{rytmický sprievod,} v angličtine popisnejšie nazvaný
\emph{strumming pattern}. Na každú dobu robí ruka pohyb dole a hore. Napríklad pre štvordobový
takt počítame:
{\larger$$
\PluckDown{Pr}\ \PluckUp{vá}\ \PluckDown{Dru}\ \PluckUp{há}\ 
\PluckDown{Tre}\ \PluckUp{tia}\ \PluckDown{Štvr}\ \PluckUp{tá}\ 
$$}%
Sprievod je tvorený tým, pri ktorých pohyboch (ne)udierame struny, kde dáme prízvuk a podobne.

Na vrchu stránky uvádzame príklad rytmického sprievodu, viď značky v tabuľke nižšie.
Ten by mal slúžiť iba ako inšpirácia, sprievod do istej miery simuluje bicie a je na každom, ako ich
interpretovať. Sprievod nie je uvedený pri piesniach, kde sa hrá \uv{štandardný} sprievod (pozri nižsie)
alebo kde nebolo jasné, aký zvoliť.

V sprievodoch s prízvukom sa niektoré neprízvučné údery nadol zvyknú hrať iba ako ľahký dotyk
struny G. Tento štýl dáva vyznieť ladeniu struny G a dotvára charakteristický zvuk ukulele.
Nie je pevné pravidlo, ktoré údery sa majú takto hrať, častokrát to ale býva neprízvučná doba pred
prízvukom.

Označenie \textbf{swing} znamená, že pohyb rukou dole trvá dlhšie ako pohyb hore -- približne
dve tretiny doby vs. tretina doby. Swingový rytmus sa hrá veľmi bežne, označenie možno pri
niektorých piesniach chýba (dajte mi vedieť). [TOTO: swing ako default?]

Jeden pohyb ruky dole a hore väčšinou zaberá jednu dobu. Označenie \textbf{double time} znamená,
že sa sprievod hrá dvakrát rýchlešie, t.j. na pol doby. Naopak, \textbf{half time} značí dvakrát
pomalšie hranie, t.j. na dve doby.

\medskip

\def\mystrut{\vrule height 18pt depth 0pt width 0pt}
\begin{tabularx}{\linewidth}{ c X }
    \mystrut $\Down\Up$ & Úder cez všetky struny dole/hore \\ 
    \mystrut $\Separator$ & Oddeľuje logické celky sprievodu, napr. zmenu akordu \\ 
    \mystrut $\AccentDown$ & Prízvuk \\  
    \mystrut $\StaccatoDown$ & Staccato, stlm struny ihneď po údere \\
    \mystrut $\ChuckDown$ & Tlmený úder, anglicky \textit{chuck} \\
    \mystrut $\PluckDown{R}$ & Rasgueado, \uv{vejárovitý} úder cez všetky struny viacerými prstami \\    
    \mystrut $\PluckDown{E A}$ & Brnkni iba struny uvedené nad akordom. Na smere šípky nezáleží. \\
\end{tabularx}


\subsection*{Niektoré základné sprievody}

\begin{tabularx}{\linewidth}{ l X }
    \mystrut $\Down\Miss\Down\Miss\Down\Miss\Down\Miss$ & 
    Jednoduchý sprievod vhodný pre začiatočíkov, ktorý často funguje prekvapivo dobre.
    Hraj úder dole na každú dobu. \\ 

    \mystrut $\Down\Miss\Down\Up\Miss\Up\Down\Up$ &
    Najrozšírenejší, \uv{štandardný} sprievod ktorý je vhodný k väčšine piesní. Nemá ustálený názov,
    po anglicky sa používajú označenia \textit{common, island, Calypso, d-du-udu} alebo
    \textit{Old faithful strum}. Bežne sa hrá so swingom. Dve populárne variácie s prízvukom:
    $\AccentDown\Miss\Down\AccentUp\Miss\Up\AccentDown\Up$ a
    $\Down\Miss\AccentDown\Up\Miss\Up\AccentDown\Up$ \\ 

    \mystrut $\Down\Miss\AccentDown\Up\ \Down\Miss\AccentDown\Up$ & Tento sprievod znie
    skvelo na ukulele. Pri úderoch pred prízvukom sa iba zľahka dotknite G struny, akoby ste hrali:
    $\PluckDown{G}\Miss\AccentDown\Up\ \PluckDown{G}\Miss\AccentDown\Up$ \\  
\end{tabularx}


\subsection*{FAQ}

\textbf{Q: Hrám na gitaru. Môžem spevník použivať ako gitarový spevník?}

Áno, určite! V 99\% prípadov sa hrajú rovnaké akordy na gitare. Niektoré pesničky boli transponované
aby sa hrali dobre na ukulele, odporúčam naučiť sa akord \ch{Eb} (ako \ch{C} s barre na treťom pražci).
Tiež si niekedy treba dať pozor na akordy s lomítkom, keďže ukulele nemá basové struny, napríklad namiesto
\ch{C/B} sa na ukulele zvykne hrať \ch{Cmaj7}.

\textbf{Q: Niektoré akordy vyzerajú zložito. Čo mám robiť?}

Akordy sa dajú zjednodušiť. Väčina akordov je odvodených buď z durového alebo mollového akordu,
takže namiesto \ch{Cmaj7} alebo \ch{Gm6} sa dá hrať \ch{C} alebo \ch{Gm}. Akord \ch{E}, ktorý
sa ťažko chytá, sa dá nahradiť \ch{E7}. A napokon niektoré akordy ako \textbf{dim} a \textbf{sus}
sa často dajú vynechať.

\textbf{Q: Prečo citrónový?}

Pretože také je pozadie na obálke.


\subsection*{Acknowledgements}

Olin :)

\endgroup