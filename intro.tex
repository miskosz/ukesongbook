% Adjust formatting.
\begingroup % Original formatting is reset back with \endgroup
\setlength{\parindent}{0pt}
\setlength{\parskip}{\baselineskip}
\large

\section*{Úvod}

Všetkých Vás vítam v Citrónovom ukulele spevníku! Či už hráte na ukulele alebo iný nástroj,
a či ste začiatočníci alebo pokročilí, dúfam, že si v spevníku nájdete niečo pre Vás.

Tento spevník sa odlišuje od ostatných hlavne v dvoch veciach. U nás sa zvyknú uvádzať akordy
iba v prvej slohe a v ostaných si ich treba buď pamätať alebo skákať očami hore-dole, čo mne
nikdy nešlo. Tu sú \emph{akordy uvedené všade.} A možno sa už aj Vám stalo, že ste si
chceli zahrať pieseň, ale zistíte, že vlastne poznáte len jej refrén alebo časť slohy.
Alebo sprevádzate niekoho, kto ju pozná, akurát by ste potrebovali vedieť, kedy zmeniť akord.
V tomto spevníku je preto naznačené, \emph{ako dlho akordy hrať.}

A keďže hrám na ukulele, tak vznikol ukulele spevník!\footnote{Spevník bol inšpirovaný Londýnskym
spevníkom Ukulele Wednesdays, ktorý vrelo odporúčam!} Akordy sú miestami prispôsobé tak, aby sa
hrali dobre práve na tomto nástroji, ale spevník je použiteľný napríklad aj pre gitaru. Tiež dúfam,
že takto môžem aspoň trochu prispieť k formujúcej sa ukulele komunite na Slovensku.

Toľko odo mňa na úvod, začnite listovať, a hlavne uživajte si radosť z hrania! Môžete
mi dať vedieť ak sa Vám spevník páči alebo ak nájdete nejakú chybu, budem iba rád ak
by niekto chcel do spevníka prispieť.

Z lockdownu v Londýne,

{
\hspace{1cm}
Mišo

\hspace{1cm}
Január 2021
}

\bigskip

P.S.: Aj keď som sa snažil uvádzať akordy blízko originálu, \uv{správne akordy} ako také
v princípe neexistujú -- na ukulele sa len snažíme napodobniť, čo sa hrá v originále
na viacerých nástrojoch. Preto sa nebojte si akordy prispôsobiť podľa Vášho osobného muzického
cítenia :)

\section*{Použité značky}

\subsection*{Akordy}

V spevníku používame anglické značenie, ktoré je bežné u zahraničných piesní. Zápis \textbf{B, Bb}
preto označuje noty \textbf{H, B} v nemeckom značení, ktoré je rozšírené u nás.

Hudobný zápis je organizovaný do \textbf{taktov.} Každý takt pozostáva z niekoľkých
\textbf{dôb.} V spevníku značka \ch{C} značí akord, ktorý trvá celý takt, \ch{C\beats2}
značí akord trvajúci toľko dôb, koľko je čiarok. Na začiatku piesne je uvedené zápisom \beatsperchord{2},
koľko dôb má jeden takt. Keďže väčšina piesní sa hrá na štyri doby, predpis \beatsperchord{4} vynechávame.

Text obsahuje diagramy akordov pre štandardné ukulele ladenie GCEA. Pre úsporu miesta sú niekedy
vynechané nasledujúce štyri základné akordy:
\begin{center}
\smaller
\ukechord{C}
\ukechord{G}
\ukechord{F}
\ukechord{Am}
\larger
\end{center}

Na rozdiel od ostatných spevníkov sú akordy uvedené priamo v texte, text sa tak lepšie zmestí
na stránku. Pokiaľ medzi akordom a textom \textbf{nie je medzera,} slová textu sa začínajú spievať
naraz s akordom, s medzerou sa text začína spievať neskôr.\footnote{Minimálne taká bola snaha :)}

\textit{%
Poznámka pre pokročilých: Pretože ukulele nemá basové struny, akordy s lomítkom sa všeobecne
nepoužívajú. Základný tón akordu preto nemusí byť rovnaký ako čo má hrať basgitara. Napríklad
C/B alebo D/A môžu byť prepísané ako Cmaj7 alebo D.
}

\medskip

\begin{tabularx}{\linewidth}{ l X }
    \ch{C\beats2} & Hraj akord toľko dôb koľko je čiarok \\ 
    \ch{C} & Hraj akord na dĺžku jedného taktu \\
    \ch{C\rep2} & Hraj akord na dĺžku viacerých taktov \\
    \ch{\early C} & Skráť predošlý akord a predĺž tento o dobu alebo poldoby (podľa piesne)\footnote{%
    Napríklad \ch{D} \ch{\early G} môže byť realizované ako \ch{D\beats3}\ch{G\beats5}.} \\
    \ch{C-G-C} & Skratka pre hranie viacerých akordov tak, aby sa zmestili do jedného taktu \\
    \ch{N.C.} & Hudobná pauza, z anglického \textit{no chord}
\end{tabularx}


\subsection*{Rytmický sprievod}

Rukou, ktorá nedrží akord, hráme \textbf{rytmický sprievod,} v angličtine popisnejšie nazvaný
\emph{strumming pattern}. Na každú dobu robí ruka pohyb dole a hore. Napríklad pre štvordobový
takt počítame:
{\larger$$
\PluckDown{Pr}\ \PluckUp{vá}\ \PluckDown{Dru}\ \PluckUp{há}\ 
\PluckDown{Tre}\ \PluckUp{tia}\ \PluckDown{Štvr}\ \PluckUp{tá}\ 
$$}%
Sprievod je tvorený tým, pri ktorých pohyboch (ne)udierame struny, kde dáme prízvuk a podobne.

Na vrchu stránky uvádzame príklad rytmického sprievodu, viď značky v tabuľke nižšie.
Ten by mal slúžiť iba ako inšpirácia, sprievod do istej miery simuluje bicie a je na každom, ako ich
interpretovať. Sprievod nie je uvedený pri piesniach, kde sa hrá \uv{štandardný} sprievod (pozri nižsie)
alebo kde nebolo jasné, aký zvoliť.

V sprievodoch s prízvukom sa niektoré neprízvučné údery nadol zvyknú hrať iba ako ľahký dotyk
struny G. Tento štýl dáva vyznieť ladeniu struny G a dotvára charakteristický zvuk ukulele.
Nie je pevné pravidlo, ktoré údery sa majú takto hrať, častokrát to ale býva neprízvučná doba pred
prízvukom.

Označenie \textbf{swing} znamená, že pohyb rukou dole trvá dlhšie ako pohyb hore -- približne
dve tretiny doby nadol a tretinu doby nahor. Rytmus môže byť viac alebo menej swingový podľa toho,
aký je veľký tento rozdiel, pravidelný rytmus bez swingu sa nazýva \textbf{straight}. Sprievod
sa u skoro všetkých piesní sa hrá swingovo, preto toto označenie neuvádzame.

Jeden pohyb ruky dole a hore väčšinou zaberá jednu dobu. Označenie \textbf{double time} znamená,
že sa sprievod hrá dvakrát rýchlešie, t.j. na pol doby. Naopak, \textbf{half time} značí dvakrát
pomalšie hranie, t.j. na dve doby.

\medskip

\def\mystrut{\vrule height 18pt depth 0pt width 0pt}
\begin{tabularx}{\linewidth}{ c X }
    \mystrut $\Down\Up$ & Úder cez všetky struny dole/hore \\ 
    \mystrut $\Separator$ & Oddeľuje logické celky sprievodu, napr. zmenu akordu \\ 
    \mystrut $\AccentDown$ & Prízvuk \\  
    \mystrut $\StaccatoDown$ & Staccato, stlm struny ihneď po údere \\
    \mystrut $\ChuckDown$ & Tlmený úder, anglicky \textit{chuck} \\
    \mystrut $\PluckDown{R}$ & Rasgueado, \uv{vejárovitý} úder cez všetky struny viacerými prstami \\    
    \mystrut $\PluckDown{E A}$ & Brnkni iba struny uvedené nad šípkou. Na smere šípky nezáleží. \\
\end{tabularx}


\subsection*{Niektoré základné sprievody}

\begin{tabularx}{\linewidth}{ l X }
    \mystrut $\Down\Miss\Down\Miss\Down\Miss\Down\Miss$ & 
    Jednoduchý sprievod vhodný pre začiatočíkov, ktorý často funguje prekvapivo dobre.
    Hraj úder dole na každú dobu. \\ 

    \mystrut $\Down\Miss\Down\Up\Miss\Up\Down\Up$ &
    Najrozšírenejší, \uv{štandardný} sprievod ktorý je vhodný k väčšine piesní. Nemá ustálený názov,
    po anglicky sa používajú označenia \textit{common, island, Calypso, d-du-udu} alebo
    \textit{Old faithful strum}. Bežne sa hrá so swingom. Dve populárne variácie s prízvukom:
    $\AccentDown\Miss\Down\AccentUp\Miss\Up\AccentDown\Up$ a
    $\Down\Miss\AccentDown\Up\Miss\Up\AccentDown\Up$ \\ 

    \mystrut $\Down\Miss\AccentDown\Up\ \Down\Miss\AccentDown\Up$ & Pochodový sprievod, ktorý znie
    skvelo na ukulele. Pri úderoch pred prízvukom sa iba zľahka dotkni G struny, akoby ste hrali:
    $\PluckDown{G}\Miss\AccentDown\Up\ \PluckDown{G}\Miss\AccentDown\Up$ \\  
\end{tabularx}


\subsection*{FAQ}

\textbf{Q: Hrám na gitaru. Môžem spevník použivať ako gitarový spevník?}

Áno, určite! V 99\% prípadov sa hrajú rovnaké akordy na gitare. Niektoré pesničky boli transponované
aby sa hrali dobre na ukulele, odporúčam naučiť sa akord Eb (ako C s barre na treťom pražci).
Tiež si niekedy treba dať pozor na akordy s lomítkom, keďže ukulele nemá basové struny, napríklad namiesto
C/B sa na ukulele zvykne hrať Cmaj7.

\textbf{Q: Niektoré akordy vyzerajú zložito. Čo mám robiť?}

Akordy sa dajú zjednodušiť. Väčina akordov je odvodených buď z durového alebo mollového akordu,
takže namiesto Cmaj7 alebo Gm6 sa dá hrať C alebo Gm. Akord E, ktorý
sa ťažko chytá, sa dá nahradiť E7. A napokon niektoré akordy ako \textbf{dim} a \textbf{sus}
sa často dajú vynechať.

\textbf{Q: Prečo citrónový?}

Pretože najprv vzniklo pozadie na obálke, až potom názov spevníka.


\subsection*{Acknowledgements}

Ďakujem Olinovi, ktorý mi pomáhal víťaziť v bojoch s LaTeXom. Tiež ďakujem kamarátom a známym,
s ktorými som spevník testoval.

\endgroup